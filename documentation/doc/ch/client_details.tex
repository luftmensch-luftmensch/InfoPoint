\section{Client}
  Di seguito riportiamo alcuni dettagli implementativi della struttura del \textbf{Client}.

  \subsection{Struttura del client}
  La struttura del client è così strutturata:
  
  \begin{minipage}{0.45\textwidth}
    \begin{forest}
      for tree={
        font=\sffamily,
        minimum height=0.75cm,
        rounded corners=4pt,
        grow'=0,
        inner ysep=8pt,
        child anchor=west,
        parent anchor=south,
        anchor=west,
        calign=first,
        edge={rounded corners},
        edge path={
          \noexpand\path [draw, \forestoption{edge}]
          (!u.south west) +(12.5pt,0) |- (.child anchor)\forestoption{edge label};
        },
        before typesetting nodes={
          if n=1
          {insert before={[,phantom,minimum height=18pt]}}
          {}
        },
        fit=band,
        s sep=12pt,
        before computing xy={l=25pt},
      }
      [\folder{src}
        [{\folder[customOrange4]{core}}
          [{\folder[customPurple4]{builder}}]
          [{\folder[customPurple4]{config}}]
          [{\folder[customPurple4]{error}}]
          [{\folder[customPurple4]{network}}]
          [{\folder[customPurple4]{prefs}}]
          [{\folder[customPurple4]{validator}}]
        ]
        [{\folder[customOrange4]{model}}
          [{\folder[customPurple4]{prop}}]
        ]
        [{\folder[customOrange5]{ui}}
          [{\folder[customOrange5]{activity}}
            [{\folder[customPurple4]{auth}}
              [{\folder[customPurple4]{login}}]
              [{\folder[customPurple4]{reg}}]
            ]
            [{\folder[customPurple4]{intro}}]
          ]
          [{\folder[customPurple4]{adapters}}]
          [{\folder[customOrange5]{fragment}}]
        ]
      ]
    \end{forest}
  \end{minipage}
  \hfill%
  \hspace{1 cm}
  \begin{minipage}{0.45\textwidth}
    % Core
    \begin{center}
      {\Huge Core}

      (Contiene le componenti fondamentali alla definizione del client)
    \end{center}
    \vspace{0.5cm}

    \begin{itemize}
      \setlength\itemsep{1em}
    \item \hi{builder} \rightarrow Offre supporto alla costruzione di oggetti mediante Generics (Generic Builder);
    \item \hi{config} \rightarrow Contiene tutte le informazioni necessarie alla configurazione dell'applicativo;
    \item \hi{error} \rightarrow Offre supporto alla gestione degli errori che possono avvenire durante l'esecuzione dell'applicazione (Mette a disposizione una api per la gestione dello stack trace);
    \item \hi{network} (networking) \rightarrow Offre supporto alle operazioni che richiedono l'utilizzo della rete
    \item \hi{prefs} (preferences) \rightarrow Offre il supporto alle shared preferences in maniera semplice e sicura
    \item \hi{validator} \rightarrow Offre il supporto alla validazione di Username e Password inserite dall'utente
    \end{itemize}

    \vspace{1cm}

    % Model
    \begin{center}
      {\Huge Model}

      (Contiene le componenti necessarie alla definizione dei modelli utilizzati e loro proprietà)
    \end{center}
    \vspace{0.5cm}

    % UI
    \begin{center}
      {\Huge UI}

      (Contiene le componenti necessarie alla definizione dell’infrastruttura grafica)
    \end{center}
    \vspace{0.5cm}

    \begin{itemize}
      \setlength\itemsep{1em}
    \item \hi{activity} \rightarrow Contiene la definizione delle activity \footnotemark per \begin{itemize} \item Autenticazione, ovvero \begin{itemize} \item Login; \item Registrazione (reg) \end{itemize} \item Esposizione dell'applicazione (Intro) \end{itemize}

    \item \hi{adapters} \rightarrow Offre gli adapters \footnotemark utilizzati dalle activity per la gestione delle componenti grafiche che fanno uso di componenti oggetto
    \item \hi{fragments} \rightarrow Contiene la definizione di tutti i fragment \footnotemark presenti nell'applicazione
    \end{itemize}

  \end{minipage}

  \footnotetext{La "finestra" grafica con cui l'utente interagisce}
  \footnotetext{Un ponte per il componente grafico e gli oggetti che lo compongono}
  \footnotetext{La porzione "riusabile" della UI dell'applicazione}


  \subsection{Compilazione ed Esecuzione}
    Entrambe le operazioni sono effettuabili dall'IDE \hi{Android Studio} in seguito all'apertura del progetto \footnotemark \footnotetext{Per maggiori informazioni visitare il seguente \href{https://developer.android.com/studio/run/}{\hi{link}}}.
    
  \subsection{Analisi del codice sviluppato}
    In questa fase andremo quindi ad analizzare nello specifico i concetti chiave alla base dell'applicazione Android che fa da client.

    \begin{center} \hi{SharedPreferences} \end{center}
    Le \href{https://developer.android.com/reference/android/content/SharedPreferences}{\hi{Shared Preferences}} sono fondamentali per la memorizzazione delle informazioni all'interno di una applicazione Android. Proprio per questo motivo si è scelto di farne uso facendo affidamento alla classe wrapper \file{StorageManager} da noi così strutturata:

    \begin{lstlisting}[language={[POSIX]JAVA}, style=wnumbers]
    public class StorageManager {
        private static final String _TAG = "[StorageManager] ";
        private static final String PREF_NAME = "app_settings";
        private static StorageManager instance;
        private static SharedPreferences preferences;

        private StorageManager(@NonNull Context context) {
          preferences = context.getApplicationContext().getSharedPreferences(PREF_NAME, Context.MODE_PRIVATE);
        }

        private StorageManager(@NonNull Context context, String key) {
          preferences = context.getApplicationContext().getSharedPreferences(key, Context.MODE_PRIVATE);
        }

        public static StorageManager with(@NonNull Context context) {
            if (instance == null)
                instance = new StorageManager(context);
            return instance;
        }

        public static StorageManager with(@NonNull Context context, boolean forceInstantiation) {
            if (forceInstantiation)
                instance = new StorageManager(context);
            return instance;
        }

        public static StorageManager with(@NonNull Context context, String preferencesName) {
            if (instance == null)
                instance = new StorageManager(context, preferencesName);
            return instance;
        }

        public String read(String key, String defaultValue){return preferences.getString(key, defaultValue);}

        public Integer read(String key, Integer defaultValue){return preferences.getInt(key, defaultValue);}
        public Long read(String key, Long defaultValue){return preferences.getLong(key, defaultValue);}
        public Boolean read(String key, Boolean defaultValue){return preferences.getBoolean(key, defaultValue);}

        public boolean contains(String key){ return preferences.contains(key); }

        public <T> void write(String key, T value) {
            SharedPreferences.Editor editor = preferences.edit();

            // Retrieve the type of value using the simple name of the underlying class as given in the source code
            // avoiding using instanceof for each possible type (Supported type are String, Integer, Long, Boolean)
            switch (value.getClass().getSimpleName()) {
                case "String" -> editor.putString(key, (String) value);
                case "Integer" -> editor.putInt(key, (Integer) value);
                case "Long" -> editor.putLong(key, (Long) value);
                case "Boolean" -> editor.putBoolean(key, (Boolean) value);
                default -> Log.d(_TAG, "Unknown type <" + value.getClass().getSimpleName() + "> was given");
            }
            editor.apply();
        }

        public void clear() { preferences.edit().clear().apply(); }
    }
    \end{lstlisting}

    \begin{center} \hi{Modelli} \end{center}

    Rappresentazione ad Oggetti del concetto di Utente ed Opere e le loro rispettive proprietà.

    Vengono utilizzati come contenitori di conversione dei dati passati tra client e server. Sono così strutturati:
    

    \lstinputlisting[language={[POSIX]JAVA}, style=wnumbers, linerange={26-30, 40-40}]{../../client/app/src/main/java/com/infopoint/model/User.java}

    \lstinputlisting[language={[POSIX]JAVA}, style=wnumbers, linerange={26-32, 62-62}]{../../client/app/src/main/java/com/infopoint/model/ArtWork.java}


    \begin{center} \hi{Properties} \end{center}
    Per una migliore gestione dei Models si è scelto di utilizzare delle classi di supporto.

    In particolare:

    \begin{itemize}
      \item \file{ArtworkUtil} $\rightarrow$ Ha il compito di gestire tutte le opere sfruttando l'uso delle Shared Preferences mediante la classe \file{StorageManager}
      \item \file{UserType} $\rightarrow$ Permette di descrivere il livello dell'utente loggato
    \end{itemize}

    Di seguito riportiamo la loro struttura:

    \lstinputlisting[language={[POSIX]JAVA}, style=wnumbers, linerange={37-39, 41-46, 49-56}]{../../client/app/src/main/java/com/infopoint/model/properties/ArtworkUtil.java}

    \lstinputlisting[language={[POSIX]JAVA}, style=wnumbers, firstlineandnumber=24]{../../client/app/src/main/java/com/infopoint/model/properties/UserType.java}

    \begin{center} \hi{Ui} \end{center}
    Come qualsiasi applicazione che si rispetti, si è voluto porre un focus importante sulla produzione di una interfaccia semplice e intuitiva.

    A questo si aggiunge una serie di funzioni di supporto per:
    \begin{itemize}
      \item Autenticazione;
      \item Recupero delle informazioni e loro manipolazione;
      \item Operazioni di supporto all'utente.
      \item Controllo dell'accesso alla rete (fondamentale per il corretto funzionamento);
    \end{itemize}

    Di queste operazioni riportiamo alcuni estratti che ritieniamo fondamentali:

    \begin{center} \hi{Controllo della presenza della rete} \end{center}

    \lstinputlisting[language={[POSIX]JAVA}, style=wnumbers, linerange={48-54}]{../../client/app/src/main/java/com/infopoint/core/networking/NetworkManager.java}

    \begin{center} \hi{Recupero delle informazioni} \end{center}

    \lstinputlisting[language={[POSIX]JAVA}, style=wnumbers, linerange={57-115}]{../../client/app/src/main/java/com/infopoint/core/networking/NetworkManager.java}

    \begin{center} \hi{Operazioni per l'utente \footnotemark \footnotetext{Ovvero registrazione, accesso ed eliminazione dell'account}} \end{center}

    \lstinputlisting[language={[POSIX]JAVA}, style=wnumbers, linerange={117-186}]{../../client/app/src/main/java/com/infopoint/core/networking/NetworkManager.java}

    Vista la natura di queste richieste, tutte le operazioni vengono eseguite da un thread \footnotemark \footnotetext{Cerchiamo di evitare a tutti i costi di effettuari operazioni bloccanti, rendendo in questo modo l'applicazione irresponsive per tutta la durata della comunicazione}. Di seguito riportiamo un esempio concreto del loro utilizzo, preso dalla classe \file{LoginActivity}:


    \lstinputlisting[language={[POSIX]JAVA}, style=wnumbers, linerange={43-43, 103-128}]{../../client/app/src/main/java/com/infopoint/ui/activity/authentication/login/LoginActivity.java}
