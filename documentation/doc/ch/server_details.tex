\chapter{Dettagli implementativi}
  \begin{abstract}
    In questo capitolo tratteremo l'implementazione e il funzionamento delle componenti che hanno reso possibile lo sviluppo della piattaforma \hi{InfoPoint}, prestando particolare attenzione alle funzionalità richieste da programma
  \end{abstract}
  \section{Server}
    Di seguito riportiamo alcuni dettagli implementativi della struttura del \textbf{Server}. In particolare:

    \begin{itemize}
      \item Analisi della struttura del progetto;
      \item Analisi e controllo delle principali system call che il C mette a disposizione per la programmazione tramite socket\footnotemark; \footnotetext{Per un'analisi più completa si faccia riferimento al codice sorgente, come indicato nel capitolo successivo}
      \item Analisi delle funzioni di logging e controllo degli stati utilizzate;
      \item Analisi e controllo delle principali system call per la gestione del Database utilizzato;
      \item Analisi e gestione della coda dei client connessi che richiedono uno scambio di informazioni.
    \end{itemize}

    \subsection{Struttura del server}
    La struttura del server è così strutturata:

    \begin{minipage}{0.45\textwidth}
      \begin{forest}
        for tree={
          font=\sffamily,
          minimum height=0.75cm,
          rounded corners=4pt,
          grow'=0,
          inner ysep=8pt,
          child anchor=west,
          parent anchor=south,
          anchor=west,
          calign=first,
          edge={rounded corners},
          edge path={
            \noexpand\path [draw, \forestoption{edge}]
            (!u.south west) +(12.5pt,0) |- (.child anchor)\forestoption{edge label};
          },
          before typesetting nodes={
            if n=1
            {insert before={[,phantom,minimum height=18pt]}}
            {}
          },
          fit=band,
          s sep=12pt,
          before computing xy={l=25pt},
        }
        [\folder{src}
          [{\folder[customOrange4]{core}}
            [{\folder[customPurple4]{database}}]
            [{\folder[customPurple4]{message}}]
            [{\folder[customPurple4]{payload}}]
            [{\folder[customPurple4]{pool}}]
            [{\folder[customPurple4]{server}}]
          ]
          [{\folder[customOrange5]{helpers}}
            [{\folder[customPurple4]{base}}]
            [{\folder[customPurple4]{cmr}}]
            [{\folder[customPurple4]{config}}]
            [{\folder[customPurple4]{container}}]
            [{\folder[customPurple4]{handler}}]
            [{\folder[customPurple4]{logging}}]
            [{\folder[customPurple4]{utility}}]
          ]
        ]
      \end{forest}
    \end{minipage}
    \hfill%
    \hspace{1 cm}
    \begin{minipage}{0.45\textwidth}
      % Core
      \begin{center}
        {\Huge Core}

        (Contiene le componenti fondamentali alla definizione del server)
      \end{center}
      \vspace{0.5cm}

      \begin{itemize}
        \setlength\itemsep{1em}
        \item \hi{server} \rightarrow È il cuore dell’applicativo. Svolge il compito di orchestratore di tutto il programma;
        \item \hi{database} \rightarrow Contiene la struttura e la logica del gestore (handler) che ha il compito di comunicare con l’istanza del Database NOSQL MongoDB;
        \item \hi{message} \rightarrow Contiene delle funzioni wrapper utili alla comunicazione tramite socket (di cui l’applicativo fa largo uso);
        \item \hi{payload} \rightarrow Prendendo ispirazione dal protocollo HTTP facciamo uso del concetto di Payload, oggetto che rappresenta informazioni (su Utenti e Opere) che i client e il server si scambieranno.
      \end{itemize}

      \vspace{1cm}

      % Helpers
      \begin{center}
        {\Huge Helpers}

        (Contiene le componenti necessarie alla definizione dell’infrastruttura)
      \end{center}
      \vspace{0.5cm}

      \begin{itemize}
        \setlength\itemsep{1em}
        \item \hi{base} \rightarrow Contiene la definizione di \begin{itemize} \item Alias di tipi comuni (sfruttando la keyword typedef) per una più facile modifica nel tempo; \item Macro per operazioni elementari; \end{itemize}

        \item \hi{cmr} (command\_line\_runner) \rightarrow Fa da wrapper alla libreria argp.h per il parsing dei parametri passati all’eseguibile prodotto in fase di compilazione;
        \item \hi{config} \rightarrow Ha il compito di recuperare le informazioni presenti all’interno del file di configurazione (in formato ini) messo a dispozione per il settaggio delle opzioni necessarie all’esecuzione del server;
        \item \hi{pool} \rightarrow Thread-Pool utilizzata dal server per soddisfare le richieste in ingresso;
        \item \hi{utility} \rightarrow Contiene la definizione di funzioni per \begin{itemize} \item Lettura e recupero informazioni di file; \item Definizione di Regex utilizzate dal programma; \item Manipolazione di buffer (trimming e concatenazione). \end{itemize}
      \end{itemize}

    \end{minipage}

    \newpage
  \subsection{File di configurazione}
    Il server mette a disposizione la possibilità di essere configurato mediante l'uso di un file di configurazione in formato \href{https://it.wikipedia.org/wiki/File_INI}{\hi{ini}} che permette di customizzare ogni aspetto del server. Di seguito riportiamo il suo contenuto:

    \lstinputlisting[language=ini]{../../server/src/info_point.ini}

  \subsection{Compilazione}
  Tutti i file sorgente necessari alla compilazione del server sono contenuti nella directory \dir{server/src}.

  Il processo risulterà particolarmente semplificato in quanto, come ampiamente presentato nelle sezioni precedenti, faremo totale affidamento allo strumento cmake.

  Con un \qterm e una shell qualsiasi andiamo quindi ad eseguire i seguenti comandi:

  \begin{lstlisting}[language=bash]
    # Per costruire le informazioni necessarie a cmake per la futura compilazione
    cmake -S . -B build/ -DCMAKE_EXPORT_COMPILE_COMMANDS=ON
    # Per compilare il server
    cmake --build build/
    # Per l'operazione di clean up dei file generati da CMake
    cmake --build build/ --target clean
  \end{lstlisting}

  L'output ottenuto sarà da ricercare nella directory \dir{bin/} sotto il nome di \hi{InfoPointServer} \footnotemark \footnotetext{Modificabile dal file di configurazione di cmake}.

  \subsection{Esecuzione}
    L'utilizzo del server prevede la sola esecuzione dell'eseguibile in quanto è un sistema totalmente autonomo.

    Attraverso l'uso della libreria \qfunction{argp.h} il server mette a disposizione una serie di flags in modo da poter:

    \begin{itemize}
     \item Personalizzare l'esecuzione del server stesso;
     \item Avere informazioni
       \begin{itemize}
         \item Sulla versione dell'applicativo;
         \item Attraverso \hi{Help Menu} stile \href{https://en.wikipedia.org/wiki/Man_page}{man};
       \end{itemize}
    \end{itemize}

    \begin{lstlisting}[language=bash]
      # Esecuzione del server utilizzando il file di configurazione
      ./build/InfoPointServer -c src/info_point.ini
      # Esecuzione del server utilizzando facendo uso della configurazione di default (-d o -use-default-config)
      ./build/InfoPointServer -d
      # Esecuzione del server per mostrare l'help menu
      ./build/InfoPointServer --help/--usage/-?
      # Esecuzione del server per ottenere informazioni sulla versione del server
      ./build/InfoPointServer -V
    \end{lstlisting}

  \subsection{Analisi del codice sviluppato}
  In questa fase andremo quindi ad analizzare nello specifico le strutture e il codice sviluppato per l'esecuzione del server, a partire dal chiamante (\file{main}):

  \begin{lstlisting}[language={[POSIX]C}, style=wnumbers]
    #include "core/server/server.h"
    #include "helpers/config/info_point_config.h"
    #include "helpers/command_line_runner/command_line_runner.h"

    int main(int argc, char** argv) {
      // Welcome message
      fprintf(stdout, ANSI_COLOR_BMAGENTA "%s" ANSI_COLOR_RESET "\n", welcome_msg);

      // Parse command line arguments
      char* config_file = parse_command_line_arguments(argc, argv);

      // Initialize the configuration structure based on how the user invoke the server:
      // + If no config file is passed (server invoked with the -d flag) -> Populate the configuration using a default one
      // + If a config file is passed (server invoked with the -c <FILE>) -> Populate the configuration using the given <FILE>
      info_point_config* cfg = (config_file != NULL) ? provide_config(config_file) : provide_default_config();

      // Show the settings getted from the file
      cfg_pretty_print(cfg, stdout);

      server* s = init_server(cfg->ns.port, cfg->cs.max_clients, cfg->cs.max_workers, cfg->ds.username, cfg->ds.password, cfg->ds.host, cfg->ds.database_name);

      // Finally free the cfg, as we no more need it
      free(cfg);

      server_loop(s);

      destroy_server(s);
    }
  \end{lstlisting}

  \begin{center} \hi{Payload} \end{center}
  Per una più semplice gestione nel passaggio delle informazioni \hi{server} $\Longleftrightarrow$ \hi{client} si è pensato di utilizzare una classe wrapper che andasse a simulare il concetto di \href{https://it.wikipedia.org/wiki/Carico_utile_(informatica)}{\hi{Payload}}, costituita da:
  \begin{itemize}
    \item Tipo di richiesta;
    \item Contenuto effettivo del payload \footnotemark \footnotetext{Per una generalizzazione massima del contenuto si fa uso del tipo \hi{void*}}
    \item Dimensione del payload.
  \end{itemize}

  \begin{lstlisting}[language={[POSIX]C}, style=wnumbers]
    typedef enum request_t {
      char* request_type;	        // Type of the request: One of LOGIN, REGISTRATION, RETRIEVE, DELETE
      char* token;		        // Token of the caller
      char* credential_username;	// Username of the caller
      char* credential_password;	// Password of the caller
    } request_t;

    typedef struct payload_t {
      void* data; // Content of the payload
      size_t size; // Actual size
    } payload_t;
  \end{lstlisting}

  Per una semplice manipolazione del dato viene offerta la funzione \qfunction{parse\_data} che effettua il parsing in base a 2 delimitatori (personalizzabili) e una funzione wrapper \qfunction{destroy\_request} per l'eliminazione delle risorse di una richiesta appena soddisfatta.

  \begin{center} \hi{Database Handler} \end{center}

  La struct \hi{db\_handler} è da considerarsi una "classe" wrapper \footnotemark \footnotetext{Per wrapper si intende un oggetto che incapsula funzionalità di più basso livello affinchè possa semplificarne l’utilizzo ai fini produttivi} $\rightarrow$ Fa da tramite per la comunicazione con il database NOSQL MongoDB.

  Di seguito ne riportiamo una descrizione ad alto livello:

  \begin{lstlisting}[language={[POSIX]C}, style=wnumbers]
    typedef struct db_handler {
      struct instance {
        mongoc_client_pool_t* pool;	// Connection pool for multi-threaded programs
        mongoc_uri_t* uri;		// Abstraction on top of the MongoDB connection URI format
      } instance;

      // Structure representing information used to:
      // Connect & authenticate to a given mongodb instance
      // Do basic operation with the speciefied documents used by the service
      struct mongo_db_settings {
        char  name[100];		// Database name identifier
        char  database_uri[100];	// String used to connect to the mongodb instance
        char* user_collection;		// Identifier of the collection used to store & retrieve data of the users
        char* art_work_collection;	// Identifier of the collection used to store & retrieve data of the artworks
      } settings;

    } db_handler;
  \end{lstlisting}

  A supporto di questa struttura sono presenti 2 strutture che rappresentano le collezioni utilizzate dal server \footnotemark \footnotetext{Per entrambe sono presenti funzioni per l'inizializzaione e eliminazione delle risorse da queste utilizzate} :

  \begin{lstlisting}[language={[POSIX]C}, style=wnumbers]
    typedef struct art_work {
      char* name;
      char* author;
      char* date;
      char* description;
    } art_work;

    typedef struct user {
      char* id;
      char* name;
      char* password;
      char* level;
    } user;
  \end{lstlisting}


  Seguendo le \textbf{good pratices} per la programmazione di una corretta API vengono esposte una serie di metodi che vengono di seguito descritte:

  \begin{itemize}
    \item \qfunction{init\_handler} $\rightarrow$ Ha il compito di inizializzare delle risorse dell'handler ad esso associato;
    \item \qfunction{destroy\_handler} $\rightarrow$ Ha il compito di liberare le risorse dell'handler ad esso associato;
    \item \qfunction{populate\_collection} $\rightarrow$ Ha il compito di popolare la collezione delle opere con i dati prestabili;
    \item \qfunction{retrieve\_art\_works} $\rightarrow$ Ha il compito di recuperare dal database associato tutte le informazioni presenti relative alle opere (artworks);
    \item \qfunction{is\_present} $\rightarrow$ Ha il compito di controllare se il documento specificato sia presente o meno nella collezione specificata;
    \item \qfunction{insert\_single} $\rightarrow$ Ha il compito di inserire il documento specificato nella collezione specificata controllando lo stato dell'operazione;
    \item \qfunction{delete\_single} $\rightarrow$ Speculare ad \qfunction{insert\_single} ha il compito di eliminare il documento specificato nella collezione specificata controllando lo stato dell'operazione;
    \item \qfunction{parse\_bson\_as\_artwork} $\rightarrow$ Ha il compito di convertire il documento \hi{bson\_t} (binary json format) $\Longleftrightarrow$ ad \hi{artwork} wrappata nella struttura \hi{payload\_t}
    \item \qfunction{parse\_bson\_as\_user} $\rightarrow$ Ha il compito di convertire il documento \hi{bson\_t} (binary json format) $\Longleftrightarrow$ ad \hi{user} wrappata nella struttura \hi{payload\_t}
    \item \qfunction{test\_connection} $\rightarrow$ Ha il compito di testare se la connessione al database definito sia valida o meno.
  \end{itemize}

  \begin{center} \hi{Thread Pool} \end{center}
  Per una gestione concorrente e parallela il server fa uso di una \href{https://it.wikipedia.org/wiki/Thread_pool}{\hi{Thread Pool}} che attende che uno specifico lavoro venga inserito nella coda di task da eseguire. La struttura è così definita:

  \begin{lstlisting}[language={[POSIX]C}, style=wnumbers]
  typedef struct thread_pool_t {
    pthread_t* threads;		// Array of threads spawned by the thread pool
    size_t threads_alive;	// N° of threads currently active

    pthread_mutex_t lock;	// Mutex of the thread pool
    pthread_cond_t signal;	// Conditional variable for the thread pool
    queue_t* queue;		// Queue containing all the tasks that has to be executed

    bool active;		// Control switch for the worker threads
    bool on_shutdown;		// Control switch for the worker threads in order to signal shutdown
    db_handler* handler;   // Database communication handler

  } thread_pool;
  \end{lstlisting}
  Analogamente alle altre strutture anche per la Thread Pool vengono offerti dei metodi di supporto:

  \begin{itemize}
    \item \qfunction{init\_thread\_pool} $\rightarrow$ Ha il compito di inizializzare delle risorse della thread pool ad esso associato;
    \item \qfunction{destroy\_thread\_pool} $\rightarrow$ Ha il compito di liberare le risorse della thread pool ad esso associato;
    \item \qfunction{submit\_task} $\rightarrow$ Ha il compito di aggiungere alla coda di lavori della thread pool un nuovo task da eseguire;
    \item \qfunction{execute\_task} $\rightarrow$ È la funzione eseguita da tutti i threads spawnati dalla Thread Pool;
    \item \qfunction{handle\_request} $\rightarrow$ Ha il compito di gestire le richieste ricevute dai vari client \footnotemark \footnotetext{Validazione dei dati inviati, contatto con il database in esecuzione, $\dots$ } ;
  \end{itemize}

  \begin{center} \hi{Message} \end{center}
  Mette a disposizione 2 funzioni utili per un rapido invio/ricezione di dati mediante socket.

  \begin{center} \hi{Server} \end{center}
  È l'orchestratore di tutto l'applicativo. Fa da wrapper per tutte le strutture sopra citate, rendendo più snella e semplice la gestione dell'intera struttura del server. È così definito:

  \begin{lstlisting}[language={[POSIX]C}, style=wnumbers]
  typedef struct server {
    ssize_t socket;
    struct sockaddr_in transport;
    thread_pool* pool;
  } server;
  \end{lstlisting}

  Anche in questo caso sono presenti dei metodi supporto all'utilizzo della struttura:

  \begin{itemize}
    \item \qfunction{init\_server} $\rightarrow$ Ha il compito di inizializzare delle risorse del server ad esso associato;
    \item \qfunction{destroy\_server} $\rightarrow$ Ha il compito di liberare le risorse del server ad esso associato;
    \item \qfunction{server\_loop} $\rightarrow$ Ha il compito di inizializzare l'handler per la terminazione "gracefully" del programma e fare da dispatcher per l'arrivo di nuove richieste di connessione;
  \end{itemize}

  Finito l'analisi della struttura centrale del server passiamo ora all'analisi delle strutture di supporto di cui le prime fanno uso.

  \begin{center} \hi{Base} \end{center}
  Contiene la definizione di:
  \begin{itemize}
    \item Macro;
    \item Abbreviazioni di tipi.
  \end{itemize}

  \begin{center} \hi{Command Line Runner} \end{center}
  Classe wrapper per la libreria \qfunction{argp.h}

  \begin{center} \hi{Config} \end{center}
  Classe utility che:

  \begin{enumerate}
    \item Effettua il parsing del file di configurazione (se specificato);
    \item Costruisce la configurazione in base ai dati ottenuti.
  \end{enumerate}

  \begin{center} \hi{Container} \end{center}

  Contiene la definizione e la struttura della \hi{queue} \footnotemark \footnotetext{Per ovvi motivi risulta comodo utilizzare una struttura di tipo FIFO (First In - First Out)} utilizzato dalla thread pool per la gestione dei task. È così definita

  \begin{lstlisting}[language={[POSIX]C}, style=wnumbers]
  typedef struct node {
    struct node* next;	// Reference to the next node
    void* data;		// Content of the node
  } node_t;

  typedef struct queue {
    node_t* head;	// Reference to the head of the queue
    node_t* tail;	// Reference to the tail of the queue
    size_t size;	// Size of the queue
  } queue_t;
  \end{lstlisting}

  Per un più semplice utilizzo la struttura offre i seguenti metodi di supporto:

  \begin{itemize}
    \item \qfunction{init\_queue} $\rightarrow$ Ha il compito di inizializzare delle risorse della queue ad essa associata;
    \item \qfunction{destroy\_queue} $\rightarrow$ Ha il compito di liberare le risorse della queue ad essa associata;
    \item \qfunction{enqueue} $\rightarrow$ Ha il compito di inserire un nodo all'interno della queue;
    \item \qfunction{enqueue} $\rightarrow$ Ha il compito di recuperare un nodo all'interno della queue;
    \item \qfunction{retrieve\_queue\_size} $\rightarrow$ Ha il compito di recuperare la dimensione corrente della queue;
    \item \qfunction{is\_empty} $\rightarrow$ Ha il compito di controllare se la queue sia vuota o meno;
  \end{itemize}


  \begin{center} \hi{Handler} \end{center}
  Contiene la definizione per l'inizializzazione dell'handler utilizzato dal server per la terminazione in caso di ricezione di un determinato segnale (in questo caso si è utilizzato il segnale \hi{SIGINT}).

  \begin{center} \hi{Logging} \end{center}
  Contiene la definizione delle funzioni necessarie per il logging delle informazioni, che permea in tutto il progetto.

  Si noti come si fa largo uso di funzioni variadiche, permettendo una struttura dinamica del logging.

  È presente inoltre anche un messaggio di benvenuto generato mediante l'utility \hi{xxd}

  \begin{center} \hi{Utility} \end{center}
  Contiene funzioni di supporto generiche utilizzate all'interno del progetto, tra cui:

  \begin{itemize}
    \item Funzioni di \href{https://it.wikipedia.org/wiki/Sequenziamento}{\hi{trimming}};
    \item Validazione di \hi{regex};
    \item Operazioni su file;
    \item Operazioni su "stringhe";
  \end{itemize}
  \newpage
