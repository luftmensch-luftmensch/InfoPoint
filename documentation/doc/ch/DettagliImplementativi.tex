\chapter{Dettagli implementativi}
  \begin{abstract}
    In questo capitolo tratteremo l'implementazione e il funzionamento delle componenti che hanno reso possibile lo sviluppo della piattaforma \hi{InfoPoint}, prestando particolare attenzione alle funzionalità richieste da programma
  \end{abstract}
  \section{Server}
    Di seguito riportiamo alcuni dettagli implementativi della struttura del \textbf{Server}. In particolare:

    \begin{itemize}
      \item Analisi della struttura del progetto;
      \item Analisi e controllo delle principali system call che il C mette a disposizione per la programmazione tramite socket\footnotemark; \footnotetext{Per un'analisi più completa si faccia riferimento al codice sorgente, come indicato nel capitolo successivo}
      \item Analisi delle funzioni di logging e controllo degli stati utilizzate;
      \item Analisi e controllo delle principali system call per la gestione del Database utilizzato.
    \end{itemize}

    \subsection{Struttura del server}
    La struttura del server è così strutturata:


    \begin{minipage}{0.45\textwidth}
      \begin{forest}
        for tree={
          font=\sffamily,
          minimum height=0.75cm,
          rounded corners=4pt,
          grow'=0,
          inner ysep=8pt,
          child anchor=west,
          parent anchor=south,
          anchor=west,
          calign=first,
          edge={rounded corners},
          edge path={
            \noexpand\path [draw, \forestoption{edge}]
            (!u.south west) +(12.5pt,0) |- (.child anchor)\forestoption{edge label};
          },
          before typesetting nodes={
            if n=1
            {insert before={[,phantom,minimum height=18pt]}}
            {}
          },
          fit=band,
          s sep=12pt,
          before computing xy={l=25pt},
        }
        [\folder{src}
          [{\folder[customOrange4]{core}}
            [{\folder[customPurple4]{database}}]
            [{\folder[customPurple4]{message}}]
            [{\folder[customPurple4]{payload}}]
            [{\folder[customPurple4]{pool}}]
            [{\folder[customPurple4]{server}}]
          ]
          [{\folder[customOrange5]{helpers}}
            [{\folder[customPurple4]{base}}]
            [{\folder[customPurple4]{cmr}}]
            [{\folder[customPurple4]{config}}]
            [{\folder[customPurple4]{container}}]
            [{\folder[customPurple4]{handler}}]
            [{\folder[customPurple4]{logging}}]
            [{\folder[customPurple4]{utility}}]
          ]
        ]
      \end{forest}
    \end{minipage}
    \hfill%
    \hspace{1 cm}
    \begin{minipage}{0.45\textwidth}
      % Core
      \begin{center}
        {\Huge Core}

        (Contiene le componenti fondamentali alla definizione del server)
      \end{center}
      \vspace{0.5cm}

      \begin{itemize}
        \setlength\itemsep{1em}
        \item \hi{server} \rightarrow È il cuore dell’applicativo. Svolge il compito di orchestratore di tutto il programma;
        \item \hi{database} \rightarrow Contiene la struttura e la logica del gestore (handler) che ha il compito di comunicare con l’istanza del Database NOSQL MongoDB;
        \item \hi{message} \rightarrow Contiene delle funzioni wrapper utili alla comunicazione tramite socket (di cui l’applicativo fa largo uso);
        \item \hi{payload} \rightarrow Prendendo ispirazione dal protocollo HTTP facciamo uso del concetto di Payload, oggetto che rappresenta informazioni (su Utenti e Opere) che i client e il server si scambieranno.
      \end{itemize}

      \vspace{1cm}

      % Helpers
      \begin{center}
        {\Huge Helpers}

        (Contiene le componenti necessarie alla definizione dell’infrastruttura)
      \end{center}
      \vspace{0.5cm}

      \begin{itemize}
        \setlength\itemsep{1em}
        \item \hi{base} \rightarrow Contiene la definizione di \begin{itemize} \item Alias di tipi comuni (sfruttando la keyword typedef) per una più facile modifica nel tempo; \item Macro per operazioni elementari; \end{itemize}

        \item \hi{cmr} (command\_line\_runner) \rightarrow Fa da wrapper alla libreria argp.h per il parsing dei parametri passati all’eseguibile prodotto in fase di compilazione;
        \item \hi{config} \rightarrow Ha il compito di recuperare le informazioni presenti all’interno del file di configurazione (in formato ini) messo a dispozione per il settaggio delle opzioni necessarie all’esecuzione del server;
        \item \hi{pool} \rightarrow Thread-Pool utilizzata dal server per soddisfare le richieste in ingresso;
        \item \hi{utility} \rightarrow Contiene la definizione di funzioni per \begin{itemize} \item Lettura e recupero informazioni di file; \item Definizione di Regex utilizzate dal programma; \item Manipolazione di buffer (trimming e concatenazione). \end{itemize}
      \end{itemize}

    \end{minipage}

    \newpage
  \section{Client}
    Di seguito riportiamo alcuni dettagli implementativi della struttura del Client.
    % \newpage

    \begin{minipage}{0.45\textwidth}
    \begin{forest}
      for tree={
        font=\sffamily,
        minimum height=0.75cm,
        rounded corners=4pt,
        grow'=0,
        inner ysep=8pt,
        child anchor=west,
        parent anchor=south,
        anchor=west,
        calign=first,
        edge={rounded corners},
        edge path={
          \noexpand\path [draw, \forestoption{edge}]
          (!u.south west) +(12.5pt,0) |- (.child anchor)\forestoption{edge label};
        },
        before typesetting nodes={
          if n=1
          {insert before={[,phantom,minimum height=18pt]}}
          {}
        },
        fit=band,
        s sep=12pt,
        before computing xy={l=25pt},
      }
      [\folder{src}
        [{\folder[customOrange4]{core}}
          [{\folder[customPurple4]{builder}}]
          [{\folder[customPurple4]{config}}]
          [{\folder[customPurple4]{error}}]
          [{\folder[customPurple4]{network}}]
          [{\folder[customPurple4]{prefs}}]
        ]
        [{\folder[customOrange4]{model}}
          [{\folder[customPurple4]{prop}}]
        ]
        [{\folder[customOrange5]{ui}}
          [{\folder[customOrange5]{activity}}
            [{\folder[customPurple4]{auth}}
              [{\folder[customPurple4]{login}}]
              [{\folder[customPurple4]{reg}}]
            ]
            [{\folder[customPurple4]{intro}}]
          ]
          [{\folder[customPurple4]{adapters}}]
          [{\folder[customOrange5]{fragment}}]
        ]
      ]
    \end{forest}
    \end{minipage}
    \hfill%
    \hspace{1 cm}
    \begin{minipage}{0.45\textwidth}
      % Core
      \begin{center}
        {\Huge Core}

        (Contiene le componenti fondamentali alla definizione del client)
      \end{center}
      \vspace{0.5cm}

      \begin{itemize}
        \setlength\itemsep{1em}
        \item \hi{builder} \rightarrow Offre supporto alla costruzione di oggetti mediante Generics (Generic Builder);
        \item \hi{config} \rightarrow Contiene tutte le informazioni necessarie alla configurazione dell'applicativo;
        \item \hi{error} \rightarrow Offre supporto alla gestione degli errori che possono avvenire durante l'esecuzione dell'applicazione (Mette a disposizione una api per la gestione dello stack trace);
        \item \hi{network} (networking) \rightarrow Offre supporto alle operazioni che richiedono l'utilizzo della rete
        \item \hi{prefs} (preferences) \rightarrow Offre il supporto alle shared preferences in maniera semplice e sicura
      \end{itemize}

      \vspace{1cm}

      % Model
      \begin{center}
        {\Huge Model}

        (Contiene le componenti necessarie alla definizione dei modelli utilizzati e loro proprietà)
      \end{center}
      \vspace{0.5cm}

      % UI
      \begin{center}
        {\Huge UI}

        (Contiene le componenti necessarie alla definizione dell’infrastruttura grafica)
      \end{center}
      \vspace{0.5cm}

      \begin{itemize}
        \setlength\itemsep{1em}
        \item \hi{activity} \rightarrow Contiene la definizione delle activity \footnotemark \footnotetext{La "finestra" grafica con cui l'utente interagisce} per \begin{itemize} \item Autenticazione, ovvero \begin{itemize} \item Login; \item Registrazione (reg) \end{itemize} \item Esposizione dell'applicazione (Intro) \end{itemize}

        \item \hi{adapters} \rightarrow Offre gli adapters \footnotemark \footnotetext{Un ponte per il componente grafico e gli oggetti che lo compongono} utilizzati dalle activity per la gestione delle componenti grafiche che fanno uso di componenti oggetto
        \item \hi{fragments} \rightarrow Contiene la definizione di tutti i fragment \footnotemark \footnotetext{La porzione "riusabile" della UI dell'applicazione} presenti nell'applicazione
      \end{itemize}

    \end{minipage}

    
