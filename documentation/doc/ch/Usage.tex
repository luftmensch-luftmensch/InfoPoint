\chapter{Guida all'uso}
  \begin{abstract}
    <TODO>
  \end{abstract}
  \section{Guida al Server}
    \subsection{Funzionalità}
      Il Sistema, deve offrire, una serie di funzionalità:
      \begin{itemize}
        \item Possibilità di connessione concorrente;
        \item Possibilità di potersi registrare alla piattaforma\footnotemark \footnotetext{Le credenziali vengono salvate facendo uso di un Database, che risulta molto più affidabile di un semplice file di testo};
        \item Possibilità di usufruire dei contenuti in base alla tipologia di utente, in modo da permette un focus diverso in base alle sue caratteristiche\footnotemark \footnotetext{Ricordiamo che il bacino degli utenti che possono fare uso del sistema può variare da scolaresche, famiglie o esperti} ;
      \end{itemize}

    \subsection{Scelte implementative}
        Seguendo il concetto del \emph{DIVIDE ET IMPERA}\footnotemark \footnotetext{Metodologia per la risoluzione di problemi $\rightarrow$ Il problema viene diviso in sottoproblemi più semplici e si continua fino a ottenere problemi facilmente risolvibili. Combinando le soluzioni ottenute si risolve il problema originario.} si è scelto di spezzare le varie funzionalità che vengono messe a disposizione per rendere il codice facilmente manutenibile ed evitare lo stato di codice monolitico\footnotemark \footnotetext{Che risulta notoriamente più difficile da gestire e modificare nel tempo}.
    \subsection{Tecnologie e strumenti utilizzati}
      Per una migliore gestione del Sistema, si è fatto uso di una serie di strumenti.

      Durante lo sviluppo si è fatto uso dell'utility \hi{cmake}\footnotemark \footnotetext{Per maggiori informazioni visitare il seguente \href{https://cmake.org/}{sito}} , tool modulare che permette la generazione di un \hi{Makefile}\footnotemark \footnotetext{Che contiene tutte le direttive utilizzate dall'utility make, che ne permettono la corretta compilazione} .

      Per la fase di deploy invece si è fatto uso di \hi{docker}, tool che permette l'esecuzione di programmi in maniera containerizzata.

      Come sperato, la combinazione di questi tool ha permesso un passaggio immediato da una situazione di esecuzione locale (di debug) ad una 

      Come sperato, avendo adottato entrambe le strategie non si sono riscontrati problemi durante il passaggio da un ambiente locale (di testing) a uno decentralizzato (in produzione).
    \subsection{Memorizzazione dei dati}
        Per essere sempre in linea con le nuove tendenze e tecnologie si è scelto di abbandonare il classico approccio basato su un collegamento ad una base di dati relazionale, preferendo un approccio di tipo \hi{NOSQL}\footnotemark \footnotetext{Che a differenza dei classici DBMS relazioni (che offrono un approccio \texttt{relazione} ai dati) offre un approccio al documento, rendendo il design più semplice} , che offre una maggiore elasticità e scalabilità nel tempo.
    \newpage

  \section{Guida al Client}
    \subsection{Primo avvio}
    \subsection{Post registrazione}
    \subsection{Memorizzazione delle informazioni}
    \subsection{Modelli di dominio}
      \begin{center}
        \hi{Class Diagram}
      \end{center}

      \begin{center}
        \hi{Sequence Diagram}
      \end{center}
    \newpage
