\chapter{Guida all'uso}
  \begin{abstract}
    Come detto \hi{InfoPoint} è suddiviso nelle 2 componenti di \hi{server} e \hi{client}. Di seguito riportiamo, per entrambi, l'analisi e le scelte effettuate durante il loro sviluppo
  \end{abstract}

  \section{Guida al Server}
    \subsection{Funzionalità}
      Il Sistema, deve offrire, una serie di funzionalità:
      \begin{itemize}
        \item Possibilità di connessione concorrente;
        \item Possibilità di potersi registrare alla piattaforma\footnotemark \footnotetext{Le credenziali vengono salvate facendo uso di un Database, che risulta molto più affidabile di un semplice file di testo};
        \item Possibilità di usufruire dei contenuti in base alla tipologia di utente, in modo da permette un focus diverso in base alle sue caratteristiche\footnotemark \footnotetext{Ricordiamo che il bacino degli utenti che possono fare uso del sistema può variare da scolaresche, famiglie o esperti} ;
      \end{itemize}

    \subsection{Scelte implementative}
        Seguendo il concetto del \emph{DIVIDE ET IMPERA}\footnotemark \footnotetext{Metodologia per la risoluzione di problemi $\rightarrow$ Il problema viene diviso in sottoproblemi più semplici e si continua fino a ottenere problemi facilmente risolvibili. Combinando le soluzioni ottenute si risolve il problema originario.} si è scelto di spezzare le varie funzionalità che vengono messe a disposizione per rendere il codice facilmente manutenibile ed evitare lo stato di codice monolitico\footnotemark \footnotetext{Che risulta notoriamente più difficile da gestire e modificare nel tempo}.

        In particolare l'implementazione fa ampiamente uso di codice sviluppato in \hi{C POSIX}, che permette l'utilizzo di funzionalità
        e \hi{system call}\footnotemark su cui si basa l'intera code-base: epoll, send, recv, socket, nanosleep, threads, mutex, semaphores \dots

        \footnotetext{Una chiamata si definisce, appunto, \hi{di sistema}, quando fa uso di servizi e funzionalità a livello kernel del sistema operativo in uso.}
    \subsection{Tecnologie e strumenti utilizzati}
      Per una migliore gestione del Sistema, si è fatto uso di una serie di strumenti.

      Durante lo sviluppo si è fatto uso dell'utility \hi{cmake}\footnotemark \footnotetext{Per maggiori informazioni visitare il seguente \href{https://cmake.org/}{sito}} , tool modulare che permette la generazione di un \hi{Makefile}\footnotemark

      automatizzato.

      \footnotetext{Che contiene tutte le direttive utilizzate dall'utility make, che ne permettono la corretta compilazione}

      Per la fase di deploy invece si è fatto uso di \hi{docker}, tool che permette l'esecuzione di programmi in maniera containerizzata.

      Come sperato, la combinazione di questi tool ha permesso un passaggio immediato da una situazione di esecuzione locale (di debug) ad una 

      Come sperato, avendo adottato entrambe le strategie non si sono riscontrati problemi durante il passaggio da un ambiente locale (di testing) a uno decentralizzato (in produzione).
    \subsection{Memorizzazione dei dati}
    Per essere sempre in linea con le nuove tendenze e tecnologie si è scelto di abbandonare il classico approccio basato su un collegamento ad una base di dati relazionale, preferendo un approccio di tipo \hi{NOSQL}\footnotemark , \footnotetext{Che a differenza dei classici DBMS relazioni (che offrono un approccio \texttt{relazione} ai dati) offre un approccio al documento, rendendo il design più semplice} che offre una maggiore elasticità \footnotemark \footnotetext{Per loro natura infatti sono pensati per lavorare in situazioni di alto carico di lavoro pur mantenendo basse latenze}
    e scalabilità\footnotemark nel tempo.

    \footnotetext{Per la loro natura offrono una forte resilienza a situazioni di fault, considerando il focus scelto (AP) nel \href{https://www.ibm.com/it-it/cloud/learn/cap-theorem}{\hi{CAP}} (Consistency-Availability-Partition Tolerance) }
    \newpage
  \section{Guida al Client}
    Prestando particolare attenzione alla semplicità di utilizzo che un'applicazione mobile deve garantire si è scelto di fare uso di un’interfaccia snella e lineare.
    \subsection{Primo avvio}
      All'avvio all'utente è richiesto di registrarsi o accedere alla piattaforma, potendo scegliere tra 3 diverse metodologie:
      \begin{itemize}
        \item Sblocco mediante \hi{username} e \hi{password};
        \item Sblocco con \hi{impronta digitale} (se supportato);
        \item Sblocco con \hi{riconoscimento facciale} (se supportato);
      \end{itemize}
    \subsection{Post registrazione}
      In seguito ad una corretta registrazione l'utente accede alla HomePage, nella quale ha la possibilità di visualizzare:
      \begin{itemize}
        \item L'elenco delle \hi{artwork} presenti nel museo;
        \item Il proprio profilo;
        \item L'accesso diretto alle artwork preferite.
      \end{itemize}
    \subsection{Memorizzazione delle informazioni}
    In linea con il design pattern \hi{MVVM} \footnotemark \footnotetext{(Model–view–viewmodel), pattern nel quale viene astratto lo stato di view (visualizzazione) e comportamento}, si è fatto uso delle \href{https://developer.android.com/reference/android/content/SharedPreferences}{\hi{SharedPreferences}} per il salvataggio e la gestione di componenti chiave quali:
      \begin{itemize}
        \item Credenziali dell'utente;
        \item Variabili d'ambiente;
        \item Variabili di stato;
      \end{itemize}
    
    \subsection{Modelli di dominio}
      \begin{center}
        \hi{Class Diagram}

            Di seguito riportiamo il diagramma delle classi di Analisi prodotto durante lo sviluppo della piattaforma InfoPoint
      \end{center}

      \begin{center}
        \hi{Sequence Diagram}

            Di seguito riportiamo il diagramma delle classi di Analisi prodotto durante lo sviluppo della piattaforma InfoPoint, in merito alla:
      \end{center}
    \newpage
