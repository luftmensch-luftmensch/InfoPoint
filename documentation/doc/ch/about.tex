\chapter{Su InfoPoint}
  \begin{abstract}
      \hi{$InfoPoint^{\copyright}$} è una piattaforma che nasce per offrire un supporto ai visitatori del museo.

      L'implementazione è suddivisa in 2 componenti che identificheremo come \hi{server} e \hi{client}

      \begin{itemize}
        \item Un backend scritto in \texttt{C} per la gestione dei dati, che fa da server, con la possibilità di essere hostato\footnotemark \footnotetext{Indica un servizio di rete che consiste nell'allocare su un server web delle pagine web di un sito web o di un'applicazione web, rendendolo così accessibile dalla rete Internet e ai suoi utenti} sia su bare metal sia in maniera containerizzata \footnotemark \footnotetext{È messa a dispozione una immagine docker plug \& play};
        \item Una applicazione \texttt{Android}, scritta in \texttt{Java} che fa da client;
      \end{itemize}
  \end{abstract}
    % Breve presentazione
  \section{Presentazione}

  \begingroup
  \phantom{\null}\noindent
    \begin{tikzpicture}[overlay, remember picture]
      \node at (current page.center) {\includegraphics[width=\pagewidth]{content/img/title.png}};
    \end{tikzpicture}
  \newpage
  \endgroup
