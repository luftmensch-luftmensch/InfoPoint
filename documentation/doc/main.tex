% TeX root = main.tex
\documentclass{./document_class}
\begin{document}
    \coverpage
    \tableofcontents

    \chapter{Su InfoPoint}
      \begin{abstract}
          \hi{$InfoPoint^{\copyright}$} è una piattaforma che nasce per offrire un supporto ai visitatori del museo.

          L'implementazione è suddivisa in 2 componenti che identificheremo come \hi{server} e \hi{client}

          \begin{itemize}
            \item Un backend scritto in \texttt{C} per la gestione dei dati, hostato\footnotemark \footnotetext{Indica un servizio di rete che consiste nell'allocare su un server web delle pagine web di un sito web o di un'applicazione web, rendendolo così accessibile dalla rete Internet e ai suoi utenti}  su una macchina virtuale offerta da Azure\footnotemark \footnotetext{Per maggiori informazioni visitare il seguente \href{https://azure.microsoft.com/it-it/services/virtual-machines}{sito}};
            \item Una applicazione \texttt{Android}, scritta in \texttt{Java} che fa da client;
          \end{itemize}
      \end{abstract}
        % Breve presentazione
      \section{Presentazione}

      \begingroup
      \phantom{\null}\noindent
        \begin{tikzpicture}[overlay, remember picture]
          \node at (current page.center) {\includegraphics[width=\pagewidth]{content/img/title.png}};
        \end{tikzpicture}
      \newpage
      \endgroup
    
    \chapter{Guida all'uso}
      \begin{abstract}
        <TODO>
      \end{abstract}
      \section{Guida al Server}
        \subsection{Funzionalità}
          Il Sistema, deve offrire, una serie di funzionalità:
          \begin{itemize}
            \item Possibilità di connessione concorrente;
            \item Possibilità di potersi registrare alla piattaforma\footnotemark \footnotetext{Le credenziali vengono salvate facendo uso di un Database, che risulta molto più affidabile di un semplice file di testo};
            \item Possibilità di usufruire dei contenuti in base alla tipologia di utente, in modo da permette un focus diverso in base alle sue caratteristiche\footnotemark \footnotetext{Ricordiamo che il bacino degli utenti che possono fare uso del sistema può variare da scolaresche, famiglie o esperti} ;
          \end{itemize}

        \subsection{Scelte implementative}
            Seguendo il concetto del \emph{DIVIDE ET IMPERA}\footnotemark \footnotetext{Metodologia per la risoluzione di problemi $\rightarrow$ Il problema viene diviso in sottoproblemi più semplici e si continua fino a ottenere problemi facilmente risolvibili. Combinando le soluzioni ottenute si risolve il problema originario.} si è scelto di spezzare le varie funzionalità che vengono messe a disposizione per rendere il codice facilmente manutenibile ed evitare lo stato di codice monolitico\footnotemark \footnotetext{Che risulta notoriamente più difficile da gestire e modificare nel tempo}.
        \subsection{Tecnologie e strumenti utilizzati}
          Per una migliore gestione del Sistema, si è fatto uso di una serie di strumenti.

          Durante lo sviluppo si è fatto uso dell'utility \hi{cmake}\footnotemark \footnotetext{Per maggiori informazioni visitare il seguente \href{https://cmake.org/}{sito}} , tool modulare che permette la generazione di un \hi{Makefile}\footnotemark \footnotetext{Che contiene tutte le direttive utilizzate dall'utility make, che ne permettono la corretta compilazione} .

          Per la fase di deploy invece si è fatto uso di \hi{docker}, tool che permette l'esecuzione di programmi in maniera containerizzata.

          Come sperato, la combinazione di questi tool ha permesso un passaggio immediato da una situazione di esecuzione locale (di debug) ad una 

          Come sperato, avendo adottato entrambe le strategie non si sono riscontrati problemi durante il passaggio da un ambiente locale (di testing) a uno decentralizzato (in produzione).
        \subsection{Memorizzazione dei dati}
            Per essere sempre in linea con le nuove tendenze e tecnologie si è scelto di abbandonare il classico approccio basato su un collegamento ad una base di dati relazionale, preferendo un approccio di tipo \hi{NOSQL}\footnotemark \footnotetext{Che a differenza dei classici DBMS relazioni (che offrono un approccio \texttt{relazione} ai dati) offre un approccio al documento, rendendo il design più semplice} , che offre una maggiore elasticità e scalabilità nel tempo.
        \newpage
        
      \section{Guida al Client}
        \subsection{Primo avvio}
        \subsection{Post registrazione}
        \subsection{Memorizzazione delle informazioni}
        \subsection{Modelli di dominio}
          \begin{center}
            \hi{Class Diagram}
          \end{center}

          \begin{center}
            \hi{Sequence Diagram}
          \end{center}
        \newpage

    \chapter{Protocollo applicativo}
      Come già indicato in precedenza abbiamo preferito il protocollo \hi{TCP} rispetto al protocollo \hi{UDP}, per la presenza di un controllo della congestione e affidabilità in termini di \texttt{invio/ricezione} di dati\footnotemark \footnotetext{Ricordiamo infatti che UDP non ha garanzie sulla trasmissione dei pacchetti, seguento la logica di best-effort} .

      Lo sviluppo dell'applicativo è stato inizialmente verticalizzato sulla creazione dello scheletro del Server, per avere un primo approccio nudo e crudo allo scambio di messaggi via \hi{socket}.

      Per avere un programma robusto e manutenibile si è fatto largo uso delle \textbf{good pratices} che questo tipo di comunicazione richiede. In particolare:
      \begin{itemize}
        \item La connessione viene aperta solo nel momento in cui devono essere \texttt{inviati/ricevuti} dati (Si evita in questo modo di tenere aperte connessioni in momenti in cui queste non vengono sfruttate);
        \item Si effettuano controlli di raggiungibilità del server lato client\footnotemark \footnotetext{Non ha senso infatti tenere aperta una connessione se non utilizzata, anzi si rischia anche di causare interruzione di servizio dovuti a timeout improvvisi};
        \item Vengono controllati i dati \texttt{inviati/ricevuti} sempre prima di compiere operazioni che possano minare il corretto funzionamento di \hi{Server} e \hi{Client} \footnotemark \footnotetext{Questo avviene anche attraverso un particolare pattern di costruzione dei dati}
        \item Vengono effettuati controlli e gestione degli stati di tutte le operazioni lato \hi{Server};
      \end{itemize}

    \chapter{Dettagli implementativi}
      \section{Server}
        \newpage
      \section{Client}
        \newpage

    \chapter{Codice sorgente sviluppato}
      Il codice sorgente prodotto durante lo sviluppo di $InfoPoint^{\copyright}$ è disponibile sulla piattaforma \emph{GitHub}, che ne ha permesso anche il versionamento.

      Di seguito riportiamo un link per il \href{https://github.com/luftmensch-luftmensch/InfoPoint/}{download}\footnotemark \footnotetext{Potrebbe essere richiesta l'autenticazione (il repository è per privacy privato)}

    \chapter{Ringraziamenti}
        Ringraziamo la professoressa \href{mailto:alessandra.rossi@unina.it}{Alessandra Rossi} per lo splendido corso, che ci ha permesso di conoscere nuove interessanti tecnologie e del supporto offertoci durante e dopo le lezioni.

    \credits
   
\end{document}
