% Created 2023-02-24 ven 17:05
% Intended LaTeX compiler: lualatex
\documentclass[letterpaper, 11pt]{article}

\usepackage{lmodern} % Ensures we have the right font
\usepackage[T1]{fontenc} % Basic font & characters selection
\usepackage[utf8]{inputenc}
\usepackage{fontspec}
% Define font family to use (Other options are: Iosevka, Source Code Pro, Ubuntu, Titillium -> see setup file)
\setmainfont{RobotoCondensed}[Extension=.ttf, UprightFont=*-Light, BoldFont=*-Regular, ItalicFont=*-LightItalic, BoldItalicFont=*-BoldItalic, Path=/home/valentino/Dropbox/fonts/]

% Tables, wrapping, and other options
%\usepackage{longtable} % This package defines the longtable environment, a multi-page version of tabular
% For a long time, the tabular environment was used to build tables.
% However, writing tables with tabular can be troublesome for beginners and really complex tables can be near impossible to write
% Also, tables built with the tabular environment have some typographical issues and, when color is used, can be misread by PDF readers.
% Therefore, more and more LaTeX users are calling to use the tblr environment from the tabularray package instead.
% Guide at https://www.latex-tables.com/ressources/tabularray.html (To merge cells cell{5}{1} = {c=2}{})
\usepackage{tabularray} % In order to get tblr env
\usepackage{wrapfig} % This makes the wrapfigure environment available
\usepackage{rotating} % Pretty obvious
\usepackage[normalem]{ulem} % underlining and strike-through
\usepackage{capt-of} % Captions outside of floats
\usepackage{graphicx} % Include images
\usepackage{amsmath, amsthm, amssymb, mathtools} % Subscript & superscript and math environments (amsmath),  Various symbols used for interpreting the entities (amssymb)
%% Memo for table setup
%% h -> Place the float here, i.e., approximately at the same point it occurs in the source text (however, not exactly at the spot)
%% t -> Position at the top of the page
%% b -> Position at the bottom of the page
%% p -> Put on a special page for floats only
%% ! -> Override internal parameters LaTeX uses for determining "good" float positions
%% H -> Places the float at precisely the location in the LaTeX code (Requires the float package)
\usepackage[table, xcdraw]{xcolor}
\usepackage{float} % Needed for table[H] override
\usepackage{listings} % Code highlighting
\usepackage{mdframed} % \usepackage[framemethod=TikZ]{mdframed} (Alternativa in caso si voglia usare Tikz come metodo)

% TCOLORBOX DEFINITION AND CUSTOMIZATION
\usepackage[most, many, breakable]{tcolorbox}
\tcbuselibrary{skins}

\newtcbtheorem[number within=section]{Definition}{}{enhanced,
before skip=2mm,after skip=2mm, colback=red!5,colframe=red!80!black,boxrule=0.5mm,
attach boxed title to top left={xshift=1cm,yshift*=1mm-\tcboxedtitleheight},
boxed title style={frame code={
\path[fill=tcbcolback]
([yshift=-1mm,xshift=-1mm]frame.north west)
arc[start angle=0,end angle=180,radius=1mm]
([yshift=-1mm,xshift=1mm]frame.north east)
arc[start angle=180,end angle=0,radius=1mm];
\path[left color=tcbcolback!60!black,right color=tcbcolback!60!black,
middle color=tcbcolback!80!black]
([xshift=-2mm]frame.north west) -- ([xshift=2mm]frame.north east)
[rounded corners=1mm]-- ([xshift=1mm,yshift=-1mm]frame.north east)
-- (frame.south east) -- (frame.south west)
-- ([xshift=-1mm,yshift=-1mm]frame.north west)
[sharp corners]-- cycle;
},interior engine=empty,
},
fonttitle=\bfseries,
label={#2},#1}{def}
\makeatletter
\newtcbtheorem{Note}{Nota}{enhanced,
breakable,
colback=white,
colframe=customAzure2!80!black,
attach boxed title to top left={yshift*=-\tcboxedtitleheight},
fonttitle=\bfseries,
title={#2},
boxed title size=title,
boxed title style={
sharp corners,
rounded corners=northwest,
colback=tcbcolframe,
boxrule=0pt,
},
underlay boxed title={%
\path[fill=tcbcolframe] (title.south west)--(title.south east)
to[out=0, in=180] ([xshift=5mm]title.east)--
(title.center-|frame.east)
[rounded corners=\kvtcb@arc] |-
(frame.north) -| cycle;
},
#1
}{def}
\makeatother
\newtcolorbox{note}[1][]{%
enhanced jigsaw,
colback=gray!20!white,%
colframe=gray!80!black,
size=small,
boxrule=1pt,
title=\textbf{\textit{Nota}},
halign title=flush center,
coltitle=black,
breakable,
drop shadow=black!50!white,
attach boxed title to top left={xshift=1cm,yshift=-\tcboxedtitleheight/2,yshifttext=-\tcboxedtitleheight/2},
minipage boxed title=3cm,
boxed title style={%
colback=white,
size=fbox,
boxrule=1pt,
boxsep=2pt,
underlay={%
\coordinate (dotA) at ($(interior.west) + (-0.5pt,0)$);
\coordinate (dotB) at ($(interior.east) + (0.5pt,0)$);
\begin{scope}
\clip (interior.north west) rectangle ([xshift=3ex]interior.east);
\filldraw [white, blur shadow={shadow opacity=60, shadow yshift=-.75ex}, rounded corners=2pt] (interior.north west) rectangle (interior.south east);
\end{scope}
\begin{scope}[gray!80!black]
\fill (dotA) circle (2pt);
\fill (dotB) circle (2pt);
\end{scope}
},
},
#1,
}
% Abbreviation for tcolorbox envs
\newcommand{\dfn}[2]{\begin{Definition*}[colbacktitle=red!75!black]{#1}{}#2\end{Definition*}} % Invoke with \dfn{Title}{CONTENT}
\newcommand{\qs}[2]{\begin{Note*}{#1}{}#2\end{Note*}} % Invoke with \qs{}{Content}
% COLOR DEFINITION
\definecolor{classTIKZcolor}{RGB}{222,222,222}
% DEFINIZIONE COLORI TIKZ
\definecolor{darkblue}{RGB}{0,60,104}
\definecolor{darkdark}{RGB}{22,22,22}
% DEFINIZIONE COLORI DA USARE PER IL CODICE
\definecolor{airforceblue}{rgb}{0.36, 0.54, 0.66}

% Definizione colori per i linguaggi da usare
\definecolor{orangered}{RGB}{239,134,64}
\definecolor{includeStatementCPP}{RGB}{148,123,155}
\definecolor{libraryStatementCPP}{RGB}{126,190,184}
\definecolor{colorMainCPP}{RGB}{190,116,67}
\definecolor{colorTypesCPP}{RGB}{188,90,69}
\definecolor{colorReservedKeywordsCPP}{RGB}{130,183,75}
\definecolor{colorLoopsCPP}{RGB}{185,176,176}
\definecolor{colorOtherKeywordsCPP}{RGB}{254,178,54}
\definecolor{headerJava}{HTML}{006b3C}
\definecolor{packageNameJavaDefinition}{HTML}{007BA7}
\definecolor{classKeywordJava}{HTML}{CD5C5C}
\definecolor{classNameJava}{HTML}{D2691E}
\definecolor{methodKeyword}{HTML}{FE6F5E}
\definecolor{constantsKeyword}{HTML}{FFA812}
\definecolor{attributesKeyword}{HTML}{9955BB}
\definecolor{testKeyword}{HTML}{1E90FF}
\definecolor{assertKeyword}{HTML}{6082B6}
\definecolor{expectedKeyword}{HTML}{29AB87}
\definecolor{nullKeyword}{HTML}{E66771}
\definecolor{variableKeyword}{HTML}{778899}
% Definizione colori definitiva
\definecolor{orangelight}{RGB}{238,162,82}
\definecolor{orange-apricot}{HTML}{FBCEB1}
\definecolor{greenlight}{RGB}{147,196,125}
\definecolor{purplelight}{RGB}{176,159,222}
\definecolor{bluelight}{RGB}{122,171,216}
\definecolor{lavanda}{HTML}{F4BBFF}

% Definizione gradiente di colori
% Azure
\definecolor{customAzure1}{HTML}{B1E7E1}
\definecolor{customAzure2}{HTML}{A1E2DB}
\definecolor{customAzure3}{HTML}{92DDD6}
\definecolor{customAzure4}{HTML}{83D8CF}
\definecolor{customAzure5}{HTML}{73D3C9}
\definecolor{customAzure6}{HTML}{64CEC3}
% Pink
\definecolor{customPink1}{HTML}{ECD7D5}
\definecolor{customPink2}{HTML}{E6C9C7}
\definecolor{customPink3}{HTML}{DFBBB9}
\definecolor{customPink4}{HTML}{D8ADAB}
\definecolor{customPink5}{HTML}{D2A09D}
\definecolor{customPink6}{HTML}{CC928F}
% Purple
\definecolor{customPurple1}{HTML}{C5C2EB} % Lavanda
\definecolor{customPurple2}{HTML}{B5B2E6}
\definecolor{customPurple3}{HTML}{A6A3E1}
\definecolor{customPurple4}{HTML}{9793DC}
\definecolor{customPurple5}{HTML}{8884D7}
\definecolor{customPurple6}{HTML}{7974D2}
% Yellow
\definecolor{customYellow1}{HTML}{EDE8AB}
\definecolor{customYellow2}{HTML}{EAE39A}
\definecolor{customYellow3}{HTML}{EAE39A}
\definecolor{customYellow4}{HTML}{E3DA78}
\definecolor{customYellow5}{HTML}{E0D667}
\definecolor{customYellow6}{HTML}{DCD156}
% Red
\definecolor{customRed1}{HTML}{FF7073}
\definecolor{customRed2}{HTML}{FF5C5F}
\definecolor{customRed3}{HTML}{FF474A}
\definecolor{customRed4}{HTML}{FF3336}
\definecolor{customRed5}{HTML}{FF1F22}
\definecolor{customRed6}{HTML}{FF0A0E}
% Orange
\definecolor{customOrange1}{HTML}{FABE75}
\definecolor{customOrange2}{HTML}{F9B562}
\definecolor{customOrange3}{HTML}{F9AC4E}
\definecolor{customOrange4}{HTML}{F8A23A}
\definecolor{customOrange5}{HTML}{F79926}
\definecolor{customOrange6}{HTML}{F69013}
% Brown
\definecolor{customBrown1}{HTML}{BCA576}
\definecolor{customBrown2}{HTML}{B59C69}
\definecolor{customBrown3}{HTML}{AE925B}
\definecolor{customBrown4}{HTML}{A48851}
\definecolor{customBrown5}{HTML}{967D4A}
\definecolor{customBrown6}{HTML}{897243}
% Green
\definecolor{customGreen1}{HTML}{D4E5B3}
\definecolor{customGreen2}{HTML}{CCE0A4}
\definecolor{customGreen3}{HTML}{C3DB95}
\definecolor{customGreen4}{HTML}{B9D585}
\definecolor{customGreen5}{HTML}{B1D076}
\definecolor{customGreen6}{HTML}{A8CB67}
\definecolor{redstrong}{RGB}{255,40,50}
\definecolor{greendark}{RGB}{144,161,106}
\definecolor{footerColor}{RGB}{0,163,243}
\definecolor{footerColorSurrounding}{RGB}{22,154,255}
\definecolor{ygroblue}{HTML}{179AFF}
% TIKZ
\usepackage{tikz}
% DEFINIZIONE CAMPI E FORME PER TIKZ
\usetikzlibrary{calc,shadows.blur,shapes,arrows,backgrounds,graphdrawing.trees, decorations.pathreplacing,positioning, arrows.meta, automata}

% DEFINIZIONE FORME GEOMETRICHE PER DIAGRAMMI
\tikzstyle{CIRCLE} = [circle, minimum width=0.8cm, minimum height=0.8cm,text centered, draw=black, fill=blue!30]
\tikzstyle{CIRCLESMALL} = [circle, minimum width=0.2cm, minimum height=0.2cm,text centered, draw=black, fill=black]
\tikzstyle{startstop} = [rectangle, rounded corners, minimum width=3cm, minimum height=1cm,text centered, draw=black, fill=red!30]
\tikzstyle{io} = [trapezium, trapezium left angle=70, trapezium right angle=110, minimum width=3cm, minimum height=1cm, text centered, draw=black, fill=blue!30]
\tikzstyle{process} = [rectangle, minimum width=3cm, minimum height=1cm, text centered, text width=3cm, draw=black, fill=orange!30]
\tikzstyle{decision} = [diamond, minimum width=3cm, minimum height=1cm, text centered, draw=black, fill=green!30]
\tikzstyle{class}=[rectangle, draw=black, text centered, anchor=north, text=black, text width=3cm, shading=axis, bottom color=classTIKZcolor,top color=white,shading angle=45]
\tikzstyle{arrow} = [thick,->,>=stealth]
% Tree forest
\usepackage[linguistics]{forest} %\usepackage{forest}


% OPTIONS FOR MDFRAMES (ENVIRONMENT)
\makeatletter
\mdfdefinestyle{@mdf@stubenv}{
leftmargin=2pt,
rightmargin=2pt,
innermargin=0pt,
outermargin=0pt,
skipabove=2pt,
skipbelow=2pt,
linewidth=1pt,
linecolor=stub@tmp!80!black,
frametitlebackgroundcolor=stub@tmp!80!black,
backgroundcolor=stub@tmp,
innertopmargin=1pt,
innerbottommargin=1pt,
innerleftmargin=1pt,
innerrightmargin=1pt,
nobreak=true}
% CREATE A DEFINITION WITH YGROBLUE COLOR
% PER UTILIZZARE LA DEFINITION CREATA -> \begin{definition}[TITOLO] Contenuto \end{definition}

% SET DEFAULT OPTION FOR CODE BLOCKS
\lstset {
frame=trBL, %frame=single
framesep=\fboxsep,
framerule=\fboxrule,
frameround=fttt,
rulecolor=\color{black},
xleftmargin=\dimexpr\fboxsep+\fboxrule,
xrightmargin=\dimexpr\fboxsep+\fboxrule,
breaklines=true,
basicstyle=\small\tt,
keywordstyle=\color{blue}\sf,
columns=flexible,
}

% Settaggio stile per i linguaggi da usare
\lstdefinestyle{CPP}{
language=C++,
backgroundcolor=\color{white},
escapeinside={`'},
numbers=left,
numbersep=15pt,
numberstyle=\tiny,
commentstyle=\color{gray},
% #include statement
keywords=[1]{\#include},
keywords=[2]{ main },
% types
keywords=[3]{int, char, short, long, float, double},
% List of reserved keywords
keywords=[4]{auto, struct , unsigned, signed, enum, register, typedef, extern, return, union, continue, goto, volatile, default, define, static},
% Loops
keywords=[5]{do, while, case, else, switch, break, for, if},
% List of other keywords
keywords=[6]{void, boolean, const, sizeof, sleep},
% Colors of the keywords:
keywordstyle=[1]\color{includeStatementCPP},
keywordstyle=[2]\color{colorMainCPP},
keywordstyle=[3]\color{colorTypesCPP},
keywordstyle=[4]\color{colorReservedKeywordsCPP},
keywordstyle=[6]\color{colorOtherKeywordsCPP}
}
\lstdefinestyle{BASH}{
language=bash,
commentstyle=\color{gray},
backgroundcolor=\color{white},
numbers=left,
numbersep=15pt,
numberstyle=\tiny,
stringstyle=\color{greendark},
commentstyle=\color{gray},
keywords=[1]{exit, print_error, fail},
keywords=[2]{printf, cut, print_ok, basename, while, usage, run},
keywords=[3]{ls, find, touch, egrep, print_info, read, done},
keywords=[4]{wc},
keywordstyle=[1]\color{redstrong},
keywordstyle=[2]\color{greenlight},
keywordstyle=[3]\color{bluelight},
keywordstyle=[4]\color{purplelight}
}
\lstdefinestyle{JAVA}{
language=Java,
backgroundcolor=\color{white},
numbers=left,
numbersep=15pt,
numberstyle=\tiny,
commentstyle=\color{gray},
stringstyle=\color{gray},
keywords=[1]{package, import, static, public, return, true},
keywords=[2]{android, Manifest, content, Context, util, Log, androidx, core, app, ActivityCompat, org, test, platform, app, InstrumentationRegistry, ext, junit, runner, runners, AndroidJUnit4, Test, RunWith, Assert, com, natour, utils, constants, Constants, persistence, LocalUser, LocalUserDbManager, java, regex, Pattern, Before},
keywords=[3]{void, boolean, int, String, while, synchronized, volatile, long, double},
keywords=[4]{class, @RunWith, super},
keywords=[5]{if, Employee, try, catch},
keywords=[6]{localUser, dbManager, checkFineLocation, checkCoarseLocation, appContext, controlloRecuperoPassword},
keywords=[7]{private},
keywords=[8]{@Test, @Before},
keywords=[9]{assertTrue, assertFalse},
keywords=[10]{expected, IllegalArgumentException},
keywords=[11]{null, false},
keywords=[12]{username, email, password, confermaPassword, pattern},
keywordstyle=[1]\color{headerJava},
keywordstyle=[2]\color{packageNameJavaDefinition},
keywordstyle=[3]\color{methodKeyword},
keywordstyle=[4]\color{classKeywordJava},
keywordstyle=[5]\color{classNameJava},
keywordstyle=[6]\color{attributesKeyword},
keywordstyle=[7]\color{constantsKeyword},
keywordstyle=[8]\color{testKeyword},
keywordstyle=[9]\color{assertKeyword},
keywordstyle=[10]\color{expectedKeyword},
keywordstyle=[11]\color{nullKeyword},
keywordstyle=[12]\color{variableKeyword},
morekeywords={*,...}
}
\lstdefinestyle{XML}{
language=XML,
backgroundcolor=\color{white},
numbers=left,
numbersep=15pt,
numberstyle=\tiny,
commentstyle=\color{gray},
keywords=[1]{article, author, title, description, text, formula, math},
keywords=[2]{dbs:,dbs:book,<dbs:description>},
keywordstyle=[1]\color{colorMainCPP},
keywordstyle=[2]\color{includeStatementCPP},
}

% Colorizing links in a nicer way.
\usepackage{hyperref} % Links
\hypersetup{colorlinks, linkcolor=black, urlcolor=blue}
%\hypersetup{pdfauthor={%a}, pdftitle={%t}, pdfkeywords={%k}, pdfsubject={%d}, pdfcreator={%c}, pdflang={%L}, breaklinks=true, colorlinks=true, linkcolor=link, urlcolor=url, citecolor=cite\n}
%Bibliography
\usepackage[backend=biber,sortcites,style=verbose-trad2]{biblatex} % maxcitenames=1, maxbibnames=3, backref=true
\bibliography{./Bibliography.bib}
\usepackage{titling}
\setlength{\droptitle}{-9em}
\setlength{\parindent}{0pt}
\setlength{\parskip}{1em}
\usepackage[stretch=10]{microtype}
\usepackage{hyphenat}
\usepackage{ragged2e}
\usepackage{subfig} % Subfigures (not needed in Org I think)
\usepackage[top=1in, bottom=1.25in, left=0.55in, right=0.55in, showframe]{geometry} % Page geometry (to show frames add showframe options)
\renewcommand{\baselinestretch}{1.15}
% Page numbering - Footer
\usepackage{fancyhdr} % Custom headers and footers
\pagestyle{fancy} % Makes all pages in the document conform to the custom headers and footers
\fancyhead{} % No page header
\renewcommand{\headrulewidth}{0pt}
\fancyfoot[L]{} % Empty left footer
\fancyfoot[C]{} % Empty center footer
\newcommand\FrameBoxR[1]{
\fcolorbox{footerColorSurrounding}{footerColor}{\makebox[3cm][r]{\textcolor{white}{\bfseries#1}}}
}
\fancyfoot[R]{\FrameBoxR{\thepage}}
\usepackage[explicit]{titlesec}

% Title customization
\pretitle{\begin{center}\fontsize{20pt}{20pt}\selectfont}
\posttitle{\par\end{center}}
\preauthor{\begin{center}\vspace{-6bp}\fontsize{14pt}{14pt}\selectfont}
\postauthor{\par\end{center}\vspace{-25bp}}
\predate{\begin{center}\fontsize{12pt}{12pt}\selectfont}
\postdate{\par\end{center}\vspace{0em}}

% Section/subsection headings:
%Section
\titlespacing\section{0pt}{2pt}{2pt} % left margin, space before section header, space after section header
%Subsection
\titlespacing\subsection{0pt}{5pt}{-2pt} % left margin, space before subsection header, space after subsection header
%Subsubsection
\titlespacing\subsubsection{0pt}{5pt}{-2pt} % left margin, space before subsection header, space after subsection header

% List spacing & options
\usepackage{enumitem}
\usepackage{fdsymbol}
\setlist{itemsep=-2pt} % or \setlist{noitemsep} to leave space around whole list
\setlist[enumerate, 1]{label=\arabic*.}
\setlist[enumerate, 2]{label=\Roman*.}
\setlist[enumerate, 3]{label=\alph*.}
\setlist[itemize, 1]{label=$\smallblacktriangleright$}
\setlist[itemize, 2]{label=$\smalldiamond$}
\setlist[itemize, 3]{label=$\smallcircle$}
% Enumeration labels
% Invoke it with: \begin{enumerate}[label=\bfseries\tiny\protect\circled{\small\arabic*}] \end{enumerate}
\newcommand*\circled[1]{\tikz[baseline=(char.base)]{\node[shape=circle,draw,inner sep=1pt] (char) {#1};}} % Numeration with number inside circle
\author{Valentino Bocchetti}
\date{}
\title{}
\begin{document}

\thispagestyle{empty}
\newgeometry{margin=0pt}
\begin{tikzpicture}[remember picture, overlay]
  \begin{scope}

    % STRUTTURA ESTERNA (ANGOLI E COLORAZIONE)
    \node[
      isosceles triangle,
      isosceles triangle apex angle=90,
      draw,
      rotate=315,
      fill=darkblue,
      minimum size =55cm] (triangoloPrimoLivelloAngoloInferiore)
    at ($(current page.south east)$)
    {};

    \node[
      isosceles triangle,
      isosceles triangle apex angle=90,
      draw,
      rotate=315,
      fill=footerColorSurrounding,
      minimum size =50cm] (triangoloSecondoLivelloAngoloInferiore)
    at ($(current page.south east)$)
    {};

    \node[
      isosceles triangle,
      isosceles triangle apex angle=90,
      draw,
      rotate=315,
      fill=darkdark,
      minimum size =35cm] (triangoloTerzoLivelloAngoloInferiore)
    at ($(current page.south east)$)
    {};

    \node[
      isosceles triangle,
      isosceles triangle apex angle=90,
      draw,
      rotate=315,
      fill=white,
      minimum size =32cm] (triangoloQuartoLivelloAngoloInferiore)
    at ($(current page.south east)$)
    {};


    \node[
      isosceles triangle,
      isosceles triangle apex angle=90,
      draw,
      rotate=135,
      fill=white,
      minimum size =31cm] (triangoloSuperiore)
    at ($(current page.north west)$)
    {};


    % STRUTTURA DEL CERCHIO E IL SUO CONTENUTO
    \node [circle, minimum size=15cm, fill=white, draw=darkblue, line width = 7pt, xshift=11cm, yshift=1cm](centro)
    at ($(current page.west)$)
    {\includegraphics[scale=.3]{./Start-Page/logo.png}};

    %\node[rectangle,
    %  draw,
    %  minimum width=4cm,
    %  minimum height=2mm,
    %  xshift= 6.42cm,
    %  yshift= -11cm,
    %  minimum height=2mm,
    %  fill = darkblue] (r) at (0,0) {};
 
    %\node[scale=1.5] at (11,-13){\itshape{\LARGE{Università degli studi di Napoli}}};
    %\node[scale=1.5] at (11,-15){\itshape{\LARGE{Federico II}}};


   %% STRUTTURA ANGOLO SINISTRO (angolo north-west)
   \node[scale=3] at (1,-1){\includegraphics[height=10pt,width=10pt]{./Start-Page/calendar.png}};
   \node[scale=2] at (6,-1){\itshape{\LARGE{A.A. 2022-2023}}};

   %% ID GRUPPO
   \node[scale=3] at (1,-4){\includegraphics[height=10pt,width=10pt]{./Start-Page/group-id.png}};
   \node[scale=1.4] at (4.3, -4.1){ID Gruppo: LSO\_2122\_23};

    
   %% STRUTTURA ANGOLO DESTRO (angolo south-east)
   \begin{scope}[node distance=15mm and 1mm]
     %% Valentino Bocchetti
     \node [scale=3,] (StudentA) at (14, -22) {\includegraphics[height=10pt,width=10pt]{./Start-Page/graduated.png}};
        \node[scale=1.2, right=of StudentA] (StudentAInfo) {\textbf{Valentino Bocchetti - N86003405}};

     %% Dario Morace
     \node [scale=3, below= of StudentA.west,anchor=west] (StudentB) {\includegraphics[height=10pt,width=10pt]{./Start-Page/graduated.png}};
        \node[scale=1.2, right=of StudentB] (StudentBInfo) {\textbf{Dario Morace - }};

     %% Lucia Brando
     \node [scale=3, below= of StudentB.west,anchor=west] (StudentC) {\includegraphics[height=10pt,width=10pt]{./Start-Page/graduated_alt.png}};
        \node[scale=1.2, right=of StudentC] (StudentCInfo) {\textbf{Lucia Brando - }};
   \end{scope}

  \end{scope}

\end{tikzpicture}

\newpage
\restoregeometry
\renewcommand*\contentsname{\hfill Indice \hfill}
\tableofcontents
\pagebreak

\noindent\rule{\textwidth}{0.5pt}
\section{Revisioni}
\label{Revisioni}
\section{Presentazione}
\label{Presentazione}
\noindent\makebox[\textwidth]{\includegraphics[width=\paperwidth]{Start-Page/title.png}}

\(InfoPoint^{\copyright}\) è un progetto che nasce per offrire un supporto ai visitatori del museo.

Questo progetto si concretizza in 2 componenti ben definite:
\begin{itemize}
\item Un backend scritto in \texttt{C} per la gestione dei dati, hostato \autocite{HOSTING} su una macchina virtuale offerta da Azure \autocite{VirtualMachines};
\item Una applicazione \texttt{Android}, scritta in \texttt{Java} che fa da client;
\end{itemize}
\section{Guida al Server}
\label{Guida al Server}
\subsection{Funzionalità}
\label{Guida al Server - Funzionalità}
Il Sistema, deve offrire, una serie di funzionalità:
\begin{itemize}
\item Possibilità di connessione concorrente;
\item Possibilità di potersi registrare alla piattaforma \autocite{RefRegistrazione};
\item Possibilità di usufruire dei contenuti in base alla tipologia di utente, in modo da permette un focus diverso in base alle sue caratteristiche \autocite{RefUtenti};
\end{itemize}
\subsection{Scelte implementative}
\label{Guida al Server - Scelte implementative}
Seguendo il concetto del \emph{DIVIDE ET IMPERA} \autocite{DIVIDE_ET_IMPERA} si è scelto di spezzare le varie funzionalità che vengono messe a disposizione per rendere il codice facilmente manutenibile ed evitare lo stato di codice monolitico \autocite{MONOLITICO}.
\subsection{Tecnologie e strumenti utilizzati}
\label{Guida al Server -}
Per una migliore gestione del Sistema, si è fatto uso di una serie di strumenti.

Durante lo sviluppo si è fatto uso dell'utility \href{https://cmake.org/}{cmake} \autocite{CMAKE}, tool modulare che permette la generazione di un Makefile \autocite{MAKEFILE}
\subsection{Memorizzazione dei dati}
\label{Guida al Server - Memorizzazione dei dati}
\section{Guida al Client}
\label{Guida al Client}
\subsection{Primo avvio}
\label{Guida al Client - Primo avvio}
\subsection{Post registrazione}
\label{Guida al Client - Post registrazione}
\subsection{Memorizzazione delle informazioni}
\label{Guida al Client - Memorizzazione delle informazioni}
\subsection{Modelli di Dominio}
\label{sec:orga47dad6}
\begin{center}
\textbf{Class Diagram}
\end{center}

\begin{center}
\textbf{Sequence Diagram}
\end{center}
\section{Protocollo applicativo}
\label{Protocollo applicativo}
Come già indicato in precedenza abbiamo preferito il protocollo \texttt{TCP} rispetto al protocollo \texttt{UDP}, per la presenza di un controllo della congestione e affidabilità in termini di \texttt{invio/ricezione} di dati \autocite{UDP}.

Lo sviluppo dell'applicativo è stato inizialmente verticalizzato sulla creazione dello scheletro del Server, per avere un primo approccio nudo e crudo allo scambio di messaggi via \texttt{socket}.

Per avere un programma robusto e manutenibile si è fatto largo uso delle \texttt{good pratices} che questo tipo di comunicazione richiede. In particolare:
\begin{itemize}
\item La connessione viene aperta solo nel momento in cui devono essere \texttt{inviati/ricevuti} dati (Si evita in questo modo di tenere aperte connessioni in momenti in cui queste non vengono sfruttate);
\item Si effettuano controlli di raggiungibilità del server lato client \autocite{RAGGIUNGIBILE};
\item Vengono controllati i dati \texttt{inviati/ricevuti} sempre prima di compiere operazioni che possano minare il corretto funzionamento di \texttt{Server} e \texttt{Client} \autocite{CONTROLLO};
\item Vengono effettuati controlli e gestione degli stati di tutte le operazioni lato \texttt{Server.}
\end{itemize}
\section{Dettagli implementativi}
\label{Dettagli implementativi}
\subsection{Server}
\label{Dettagli implementativi - Server}
\subsection{Client}
\label{Dettagli implementativi - Client}
\section{Codice sorgente sviluppato}
\label{Codice sorgente sviluppato}
Il codice sorgente prodotto durante lo sviluppo di \(InfoPoint^{\copyright}\) è disponibile sulla piattaforma \emph{GitHub}, che ne ha permesso anche il versionamento.

Di seguito riportiamo un link per il \href{https://github.com/luftmensch-luftmensch/InfoPoint/}{download} \autocite{informazioniRepository}
\section{Contributori}
\label{Contributori}
\section{Ringraziamenti}
\label{Ringraziamenti}
Ringraziamo la professoressa \href{mailto:alessandra.rossi@unina.it}{Alessandra Rossi} per lo splendido corso, che ci ha permesso di conoscere nuove interessanti tecnologie e del supporto offertoci durante e dopo le lezioni.
\end{document}